\documentclass{article}

% Packages forked from the original template
\usepackage{fancyhdr}
\usepackage{extramarks}
\usepackage{amsmath}
\usepackage{amsthm}
\usepackage{amsfonts}
\usepackage{tikz}
\usepackage[plain]{algorithm}
\usepackage{algpseudocode}

% Extra packages
\usepackage{changepage} % adjustwidth environment
\usepackage{hyperref} % href

\usetikzlibrary{automata,positioning}

% Additional commands
\def\eg{\emph{e.g., }}
\def\ie{\emph{i.e., }}
\def\cf{\emph{c.f., }}
\def\etc{\emph{etc. }}
\def\wrt{\emph{w.r.t. }}
\def\etal{\emph{et al. }}

%
% Basic Document Settings
%

\topmargin=-0.45in
\evensidemargin=0in
\oddsidemargin=0in
\textwidth=6.5in
\textheight=9.0in
\headsep=0.25in

\linespread{1.1}

\pagestyle{fancy}
\lhead{\hmwkClass: \hmwkTitle}
\rhead{\firstxmark}
\lfoot{\lastxmark}
\cfoot{\thepage}

\renewcommand\headrulewidth{0.4pt}
\renewcommand\footrulewidth{0.4pt}

\setlength\parindent{0pt}

%
% Create Problem Sections
%

\newcommand{\enterProblemHeader}[1]{
    \nobreak\extramarks{}{Exercise \arabic{#1} continued on next page\ldots}\nobreak{}
    \nobreak\extramarks{Exercise \arabic{#1} (continued)}{Exercise \arabic{#1} continued on next page\ldots}\nobreak{}
}

\newcommand{\exitProblemHeader}[1]{
    \nobreak\extramarks{Exercise \arabic{#1} (continued)}{Exercise \arabic{#1} continued on next page\ldots}\nobreak{}
    \stepcounter{#1}
    \nobreak\extramarks{Exercise \arabic{#1}}{}\nobreak{}
}

\setcounter{secnumdepth}{0}
\newcounter{partCounter}
\newcounter{homeworkProblemCounter}
\setcounter{homeworkProblemCounter}{1}
\nobreak\extramarks{Exercise \arabic{homeworkProblemCounter}}{}\nobreak{}

%
% Homework Problem Environment
%
% This environment takes an optional argument. When given, it will adjust the
% problem counter. This is useful for when the problems given for your
% assignment aren't sequential. See the last 3 problems of this template for an
% example.
%
\newenvironment{homeworkProblem}[1][-1]{
    \ifnum#1>0
        \setcounter{homeworkProblemCounter}{#1}
    \fi
    \section{Exercise \arabic{homeworkProblemCounter}}
    \setcounter{partCounter}{1}
    \enterProblemHeader{homeworkProblemCounter}
}{
    \exitProblemHeader{homeworkProblemCounter}
}

%
% Homework Details
%   - Title
%   - Due date
%   - Class
%   - Section/Time
%   - Instructor
%   - Author
%

\newcommand{\hmwkTitle}{Assignment 2}
\newcommand{\hmwkDueDate}{May 7, 2024}
\newcommand{\hmwkClass}{Continuous Optimization}
\newcommand{\hmwkAuthorName}{ \textbf{Honglu Ma} \and \textbf{Hiroyasu Akada} \and \textbf{Mathivathana Ayyappan}}

%
% Title Page
%

\title{
    \vspace{2in}
    \textmd{\textbf{\hmwkClass:\ \hmwkTitle}}\\
    \normalsize\vspace{0.1in}\small{Due\ on\ \hmwkDueDate}\\
    \vspace{3in}
}

\author{\hmwkAuthorName}
\date{}

\renewcommand{\part}[1]{\textbf{\large Part \Alph{partCounter}}\stepcounter{partCounter}\\}

%
% Various Helper Commands
%

% Useful for algorithms
\newcommand{\alg}[1]{\textsc{\bfseries \footnotesize #1}}

% For derivatives
\newcommand{\deriv}[1]{\frac{\mathrm{d}}{\mathrm{d}x} (#1)}

% For partial derivatives
\newcommand{\pderiv}[2]{\frac{\partial}{\partial #1} (#2)}

% Integral dx
\newcommand{\dx}{\mathrm{d}x}

% Alias for the Solution section header
\newcommand{\solution}{\textbf{\large Solution}}

% Probability commands: Expectation, Variance, Covariance, Bias
\newcommand{\E}{\mathrm{E}}
\newcommand{\Var}{\mathrm{Var}}
\newcommand{\Cov}{\mathrm{Cov}}
\newcommand{\Bias}{\mathrm{Bias}}

% Norm
\newcommand{\norm}[1]{\left\lVert#1\right\rVert}

% Margined Homework Subsection
\newenvironment{homeworkSubsection}[1]{%
    \subsection*{#1}%
    \begin{adjustwidth}{2.5em}{0pt}%
}{%
    \end{adjustwidth}%
}

\begin{document}

\maketitle

\pagebreak

\begin{homeworkProblem}[1]
    \begin{homeworkSubsection}{(i)}
        The gradient of $\varphi(x)$ is given by 
        \begin{align*}
            \nabla\varphi(x) 
            &= \frac{1}{2}(A + A^\top)x - b\\
            &= \frac{1}{2}(2A)x - b\tag*{$A$ is symetric}\\
            &= Ax - b\\
        \end{align*}
        Similarly, the Hessian of $\varphi(x)$ is given by
        \begin{align*}
            \nabla^2\varphi(x)
            &= (D(\nabla \varphi(x)))^\top\\
            &= (A^\top)^\top = A\\
        \end{align*}
        Suppose $\nabla \varphi(x^\star) = 0$ and $\nabla^2 \varphi(x^\star)$ is positive definite,
        then Theorem 6.9 shows that $x^\star$ is a local minimum of $\varphi(x)$.
        To meet the condition, we only need to show that $\nabla \varphi(x^\star) = 0$ 
        because $A$ is positive definite which meets the second part of the condition.
        Thus we can find the minimizer by solving $Ax = b$.
    \end{homeworkSubsection}
    \begin{homeworkSubsection}{(ii)}
        The steepest descent direction is given by
        \begin{align*}
            d^{(k)} 
            &= -\nabla \varphi(x^{(k)})\\
            &= -Ax^{(k)} + b = r^{(k)}\\
        \end{align*}
    \end{homeworkSubsection}
    \begin{homeworkSubsection}{(iii)}
        Let $g_k(\tau) = x^{(k)} + \tau r^{(k)}$ be the function that gives $x^{(k+1)}$ given $\tau$ at time step $k$.
        We can rewrite the objective function as $\underset{\tau > 0}{\min}\,\varphi(g_k(\tau))$ 
        and to find the minimizer, we can solve $\frac{\partial\varphi}{\partial\tau} = 0$.
        \begin{align*}
            \frac{\partial\varphi}{\partial\tau} 
            &= \frac{\partial g_k}{\partial\tau} \cdot \frac{\partial\varphi}{\partial g_k}\\
            &= (b - Ax^{(k)})^\top (Ag_k - b)\\
            &= -b^\top b + (Ax^{(k)})^\top b + b^\top Ag_k - (Ax^{(k)})^\top Ag_k\\
            &= -b^\top b + (Ax^{(k)})^\top b + b^\top A(x^{(k)} + \tau(b - Ax^{(k)})) - (Ax^{(k)})^\top A(x^{(k)} + \tau(b - Ax^{(k)}))\\
            &= -b^\top b + 2(Ax^{(k)})^\top b - (Ax^{(k)})^\top Ax^{(k)} + \tau (b - Ax^{(k)})^\top A (b - Ax^{(k)})\\
            &= -(b - Ax^{(k)})^\top(b - Ax^{(k)}) + \tau (b - Ax^{(k)})^\top A (b - Ax^{(k)}) = 0\\
            &\Rightarrow \tau = \frac{(b - Ax^{(k)})^\top(b - Ax^{(k)})}{(b - Ax^{(k)})^\top A (b - Ax^{(k)})}
        \end{align*}
        
        
    \end{homeworkSubsection}
\end{homeworkProblem}

\begin{homeworkProblem}[2]
    \begin{homeworkSubsection}{(i)}
        The derivative of the function $f(x)$ at $x^{(k)}$ dotted with the direction $d^{(k)}$ can be expressed as 
        \[ 
            \langle \nabla f(x^{(k)}) \; , \; d^{(k)} \rangle = 2x^{(k)} * (-1) = -2x^{(k)} < 0 
            \text{ for every } 
            x^{(k)} > 0.
        \]
        This shows that $d^{(k)}$ is a descent direction.
    \end{homeworkSubsection}
    \begin{homeworkSubsection}{(ii)}
        The descent method updates the current point $x^{(k)}$ based on the following formula:
        \[ 
            x^{(k+1)} = x^{(k)} + \tau_k d^{(k)}
        \]
        Substituting the given values, we get
        \[ 
            x^{(k+1)} = x^{(k)} + 2^{-k-1} * (-1) = x^{(k)} - 2^{-k-1}
        \]
        We need to show by induction that $x^{(k)} = 1 + 2^{-k}$ for all $k$.
        \begin{itemize}
        \item Base case ($k=0$): $x^{(0)} = 1 + 2^0 = 2$, which is true.
        
        \item Inductive step: Assume $x^{(k)} = 1 + 2^{-k}$ is true for some $k$. We need to show that $x^{(k+1)} = 1 + 2^{-(k+1)}$.
        \begin{align*}
            x^{(k+1)} &= x^{(k)} - 2^{-k-1} \\
            &= 1 + 2^{-k} - 2^{-k-1} \\
            &= 1 + 2*2^{-k-1} - 2^{-k-1} \\
            &= 1 + (2-1)*2^{-k-1} \\            
            &= 1 + 2^{-(k+1)} \\
        \end{align*}
        By induction, $x^{(k)} = 1 + 2^{-k}$ for all $k$.
        \end{itemize}
    \end{homeworkSubsection}
    \begin{homeworkSubsection}{(iii)}
        As $k$ approaches infinity, $2^{-k} \to 0$ and $1 + 2^{-k} \to 1$, showing that the sequence $x^{(k)}$ converges to 1, not 0.
    \end{homeworkSubsection}
    \begin{homeworkSubsection}{(iv)}
        $f(x)$ has its minimum at $x=0$. That is, while the sequence $x^{(k)}$ converges to 1, this is not the minimizer of $f(x)$.
        This suggests that the Wolfe’s conditions might not be satisfied. We next check if the conditions holds, which can be shown as
        \begin{align*}
            &\langle \nabla f(\overline{x^{(k)}} + \tau_{k} d^{(k)}) \; , \; d^{(k)} \rangle \leq \eta \langle \nabla f(\overline{x^{(k)}}) \; , \; d^{(k)} \rangle 
            \text{ for some } \eta \in (\gamma, 1) \\
            &\Rightarrow \langle 2 (\overline{x^{(k)}} + \tau_{k} d^{(k)}) \; , \; d^{(k)} \rangle \leq \eta \langle 2 (\overline{x^{(k)}}) \; , \; d^{(k)} \rangle \\
            &\Rightarrow \langle 2 (\overline{x^{(k)}} + \tau_{k} (-1)) \; , \; (-1) \rangle \leq \eta \langle 2 (\overline{x^{(k)}}) \; , \; (-1) \rangle \\
            &\Rightarrow -2 (\overline{x^{(k)}} + \tau_{k} (-1)) \leq -2 \eta (\overline{x^{(k)}}) \\
            &\Rightarrow \overline{x^{(k)}} - \tau_{k} \geq \eta \overline{x^{(k)}} \\       &\Rightarrow (1-\eta)\overline{x^{(k)}} \geq \tau_{k} \\
        \end{align*}
        This inequality does not hold. This is because as $k$ approaches infinity, $\tau_{k} \to 0$ but $(1-\eta)\overline{x^{(k)}} \to (1-\eta)$ that is larger than 0 since $\eta < 0$.
        Thus, the Wolfe’s conditions are not be satisfied.

    \end{homeworkSubsection}
\end{homeworkProblem}

\begin{homeworkProblem}[3]
    \begin{homeworkSubsection}{(a)}
        To show that $f$ has a global minimizer at $(a, a^2)$ we need to show that $f(a, a^2) \leq f(x_1, x_2)$ for all $(x_1, x_2)$.
        \[
            f(a, a^2) - f(x_1, x_2)
            = 0 - (a - x_1)^2 - b(x_2 - x_1^2)^2 \leq 0
        \]
    \end{homeworkSubsection}
    \begin{homeworkSubsection}{(b)}
        
    \end{homeworkSubsection}
\end{homeworkProblem}

\end{document}
