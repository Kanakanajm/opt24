\documentclass{article}

% Packages forked from the original template
\usepackage{fancyhdr}
\usepackage{extramarks}
\usepackage{amsmath}
\usepackage{amsthm}
\usepackage{amsfonts}
\usepackage{tikz}
\usepackage[plain]{algorithm}
\usepackage{algpseudocode}

% Extra packages
\usepackage{changepage} % adjustwidth environment
\usepackage{hyperref} % href
\usepackage{xcolor} % colored text

\usetikzlibrary{automata,positioning}

% Additional commands
\def\eg{\emph{e.g., }}
\def\ie{\emph{i.e., }}
\def\cf{\emph{c.f., }}
\def\etc{\emph{etc. }}
\def\wrt{\emph{w.r.t. }}
\def\etal{\emph{et al. }}

%
% Basic Document Settings
%

\topmargin=-0.45in
\evensidemargin=0in
\oddsidemargin=0in
\textwidth=6.5in
\textheight=9.0in
\headsep=0.25in

\linespread{1.1}

\pagestyle{fancy}
\lhead{\hmwkClass: \hmwkTitle}
\rhead{\firstxmark}
\lfoot{\lastxmark}
\cfoot{\thepage}

\renewcommand\headrulewidth{0.4pt}
\renewcommand\footrulewidth{0.4pt}

\setlength\parindent{0pt}

%
% Create Problem Sections
%

\newcommand{\enterProblemHeader}[1]{
    \nobreak\extramarks{}{Exercise \arabic{#1} continued on next page\ldots}\nobreak{}
    \nobreak\extramarks{Exercise \arabic{#1} (continued)}{Exercise \arabic{#1} continued on next page\ldots}\nobreak{}
}

\newcommand{\exitProblemHeader}[1]{
    \nobreak\extramarks{Exercise \arabic{#1} (continued)}{Exercise \arabic{#1} continued on next page\ldots}\nobreak{}
    \stepcounter{#1}
    \nobreak\extramarks{Exercise \arabic{#1}}{}\nobreak{}
}

\setcounter{secnumdepth}{0}
\newcounter{partCounter}
\newcounter{homeworkProblemCounter}
\setcounter{homeworkProblemCounter}{1}
\nobreak\extramarks{Exercise \arabic{homeworkProblemCounter}}{}\nobreak{}

%
% Homework Problem Environment
%
% This environment takes an optional argument. When given, it will adjust the
% problem counter. This is useful for when the problems given for your
% assignment aren't sequential. See the last 3 problems of this template for an
% example.
%
\newenvironment{homeworkProblem}[1][-1]{
    \ifnum#1>0
        \setcounter{homeworkProblemCounter}{#1}
    \fi
    \section{Exercise \arabic{homeworkProblemCounter}}
    \setcounter{partCounter}{1}
    \enterProblemHeader{homeworkProblemCounter}
}{
    \exitProblemHeader{homeworkProblemCounter}
}

%
% Homework Details
%   - Title
%   - Due date
%   - Class
%   - Section/Time
%   - Instructor
%   - Author
%

\newcommand{\hmwkTitle}{Assignment 5}
\newcommand{\hmwkDueDate}{May 28, 2024}
\newcommand{\hmwkClass}{Continuous Optimization}
\newcommand{\hmwkAuthorName}{ \textbf{Honglu Ma} \and \textbf{Hiroyasu Akada} \and \textbf{Mathivathana Ayyappan}}

%
% Title Page
%

\title{
    \vspace{2in}
    \textmd{\textbf{\hmwkClass:\ \hmwkTitle}}\\
    \normalsize\vspace{0.1in}\small{Due\ on\ \hmwkDueDate}\\
    \vspace{3in}
}

\author{\hmwkAuthorName}
\date{}

\renewcommand{\part}[1]{\textbf{\large Part \Alph{partCounter}}\stepcounter{partCounter}\\}

%
% Various Helper Commands
%

% Useful for algorithms
\newcommand{\alg}[1]{\textsc{\bfseries \footnotesize #1}}

% For derivatives
\newcommand{\deriv}[1]{\frac{\mathrm{d}}{\mathrm{d}x} (#1)}

% For partial derivatives
\newcommand{\pderiv}[2]{\frac{\partial}{\partial #1} (#2)}

% Integral dx
\newcommand{\dx}{\mathrm{d}x}

% Alias for the Solution section header
\newcommand{\solution}{\textbf{\large Solution}}

% Probability commands: Expectation, Variance, Covariance, Bias
\newcommand{\E}{\mathrm{E}}
\newcommand{\Var}{\mathrm{Var}}
\newcommand{\Cov}{\mathrm{Cov}}
\newcommand{\Bias}{\mathrm{Bias}}

% Norm
\newcommand{\norm}[1]{\left\lVert#1\right\rVert}

% Margined Homework Subsection
\newenvironment{homeworkSubsection}[1]{%
    \subsection*{#1}%
    \begin{adjustwidth}{2.5em}{0pt}%
}{%
    \end{adjustwidth}%
}

\begin{document}

\maketitle

\pagebreak

\begin{homeworkProblem}[1]
    Consider quadratic function $f(x) = \frac{1}{2}x^\top Qx + b^\top x$
    where $Q$ is a symmetric positive definite matrix.
    Using Newton's method we obtain
    \begin{align*}
        x^{(1)} &= x^{(0)} - (\nabla^2f(x^{(0)}))^{-1}\,\nabla f(x^{(0)}) \\
        &= x^{(0)} - Q^{-1}(Qx^{(0)} + b) \\
        &= x^{(0)} - Q^{-1}Qx^{(0)} + Q^{-1}b \\
        &= x^{(0)} - x^{(0)} + Q^{-1}b \\
        &= Q^{-1}b \\
    \end{align*}
    $\nabla f(x^{(1)}) = -QQ^{-1}b +b = 0 \Rightarrow x^{(1)}$ is the global minimizer
\end{homeworkProblem}

\begin{homeworkProblem}[2]
    \begin{homeworkSubsection}{(a)}
        $Q \in \mathbb{S}_{++}(n)$ with eigenvectors $(v_1, \ldots, v_n)$.
        We want to show $v_i^\top Qv_j = 0$ for any $i \neq j$.
        We know $Q = U\Lambda U^\top$ where
        \[
            U = \begin{pmatrix}
                \mid &  & \mid \\
                v_{1} & \cdots & v_{n}\\
                \mid & & \mid \\
            \end{pmatrix}
        \]
        $U$ is an orthogonal matrix i.e. $UU^\top = I$.

        For any $i \neq j$, we have
        \begin{align*}
            v_i^\top Qv_j &= v_i^\top U\Lambda U^\top v_j \\
            &= v_i^\top \begin{pmatrix}
                \mid &  & \mid \\
                v_{1} & \cdots & v_{n}\\
                \mid & & \mid \\
            \end{pmatrix}
            \begin{pmatrix}
                \lambda_1 & & 0 \\
                & \ddots & \\
                0 & & \lambda_n \\
            \end{pmatrix}
            \begin{pmatrix}
                \mid &  & \mid \\
                v_{1} & \cdots & v_{n}\\
                \mid & & \mid \\
            \end{pmatrix}^\top v_j \\
            &= \begin{pmatrix}
                0 & \cdots & v_i^\top v_i & \cdots & 0
            \end{pmatrix}
            \begin{pmatrix}
                \lambda_1 & & 0 \\
                & \ddots & \\
                0 & & \lambda_n \\
            \end{pmatrix}
            \begin{pmatrix}
                0 \\
                \vdots \\
                v_j^\top v_j \\
                \vdots \\
                0
            \end{pmatrix} \\
            &= \begin{pmatrix}
                0 & \cdots & \lambda_i v_i^\top v_i & \cdots & 0
            \end{pmatrix}
            \begin{pmatrix}
                0 \\
                \vdots \\
                v_j^\top v_j \\
                \vdots \\
                0
            \end{pmatrix} \\
            &= 0
        \end{align*}
    \end{homeworkSubsection}
    \begin{homeworkSubsection}{(b)}
        $\{d^{(0)}, \cdots, d^{(k)}\}$ are conjugate directions w.r.t. $Q$.
        We have $d_i^\top Qd_j = 0$ for any $i \neq j$.

        {\bf Proof by Contradiction:}

        Assume that $\{d^{(0)}, \cdots, d^{(k)}\}$ is not a linearly independent set.
        Then there exists a non-zero vector $d_j$ such that $d_j$ can be expressed as a linear combination of some other vectors in the set
        i.e. $d_j = \sum_{i \neq j} \delta_i d_i$. 
        Take $d_m$ that is part of the linear combination. We have
        \begin{align*}
            d_j^\top Qd_m &= \left(\sum_{i \neq j} \delta_i d_i\right)^\top Qd_m \\
            &= \delta_m d_m^\top Qd_m + \left(\sum_{i \neq j, i \neq m} \delta_i d_i\right)^\top Qd_m \\
            &= \delta_m d_m^\top Qd_m \tag*{Property of Conjugate Directions} \\
            &\neq 0 \tag*{$Q \in \mathbb{S}_{++}(n)$} \\
        \end{align*}
        which is a contradiction.
    \end{homeworkSubsection}
    \begin{homeworkSubsection}{(c)}
        For all $i = 0, \cdots, k-1$, we have
        \begin{align*}
            \langle d^{(i)}, \nabla f(x^{(k)}) \rangle &= \langle d^{(i)}, Q x^{(k)} - b \rangle\\
            &= \langle Qx^{(k)}, d^{(i)} \rangle - \langle b, d^{(i)} \rangle\\
            &= \langle Q\left( x^{(i+1)} + \sum_{j=i+1}^{k-1} \tau_j d^{(j)}\right), d^{(i)} \rangle - \langle b, d^{(i)} \rangle\\
            &= \langle Qx^{(i+1)} , d^{(i)} \rangle + \sum_{j=i+1}^{k-1} \langle \tau_j d^{(j)}, Qd^{(i)} \rangle - \langle b, d^{(i)} \rangle\\
            &= \langle Qx^{(i+1)} , d^{(i)} \rangle - \langle b, d^{(i)} \rangle \tag*{Property of Conjugate Directions}\\
            &= \langle Qx^{(i+1)} - b , d^{(i)} \rangle \\
            &= \langle  \nabla f(x^{(i+1)}), d^{(i)} \rangle \\
            &= 0 \tag*{Exact line search optimal condition} \\
        \end{align*}
    \end{homeworkSubsection}
\end{homeworkProblem}
\begin{homeworkProblem}[3]
    We expand the BFGS update formula:
    \begin{align*}
        H_{k+1} &= (I - \rho_k s^{(k)}(y^{(k)})^\top) H_k (I - \rho_k y^{(k)}(s^{(k)})^\top) + \rho_k s^{(k)}(s^{(k)})^\top\\
        &= (H_k  - \rho_k s^{(k)}(y^{(k)})^\top H_k)(I - \rho_k y^{(k)}(s^{(k)})^\top) + \rho_k s^{(k)}(s^{(k)})^\top\\
        &= H_k 
        - \rho_k\left(H_k y^{(k)}\right)(s^{(k)})^\top
        - \rho_k s^{(k)}\left((y^{(k)})^\top H_k\right)
        + \rho_k^2  y^{(k)}(s^{(k)})^\top \left(\left(H_k y^{(k)}\right)(s^{(k)})^\top\right)
        + \rho_k s^{(k)}(s^{(k)})^\top\\
    \end{align*}
    by setting parentheses as above, only matrix vector multiplications are evaluated.
\end{homeworkProblem}
\begin{homeworkProblem}[4]
    \begin{homeworkSubsection}{(a)}
        \[
            \nabla g(z) = A^\top \nabla f(x)
        \]
        \[
            \nabla^2 g(z) = A^\top \nabla^2 f(x) A
        \]
    \end{homeworkSubsection}
    \begin{homeworkSubsection}{(b)}
        Apply Newton's method to $g(z)$:
        \begin{align*}
            z^{(k+1)} &= z^{(k)} - (\nabla^2 g(z^{(k)}))^{-1} \nabla g(z^{(k)}) \\
            &= z^{(k)} - (A^\top \nabla^2 f(x) A)^{-1} A^\top \nabla f(x) \\
            &= z^{(k)} - A^{-1} \nabla^2 f(x)^{-1} A^{-\top} A^\top \nabla f(x) \\
            &= z^{(k)} - A^{-1} \nabla^2 f(x)^{-1} \nabla f(x) \\
        \end{align*}
        We have such identity
        \begin{align*}
            x^{(k+1)} &= Az^{(k+1)}+b\\
            &= A\left(z^{(k)} - A^{-1} \nabla^2 f(x)^{-1} \nabla f(x)\right) + b\\
            &= Az^{(k)} - AA^{-1} \nabla^2 f(x)^{-1} \nabla f(x) + b\\
            &= Az^{(k)} + b - \nabla^2 f(x)^{-1} \nabla f(x)\\
            &= x^{(k)} - \nabla^2 f(x)^{-1} \nabla f(x)\\
        \end{align*}
        which is the same as applying Newton's method to $f(x)$.
    \end{homeworkSubsection}
    \begin{homeworkSubsection}{(c)}
        Apply BFGS to $g(z)$:
        In case of $k = 0$, we have
        \begin{align*}
            z^{(1)} &= z^{(0)} - \tilde{H}_0\nabla g(z^{(0)})\\
            &= z^{(0)} - \tilde{H}_0 A^\top \nabla f(x)\\
            &= z^{(0)} - A^{-1} H_0 \nabla f(x) \tag*{$H_0 = A\tilde{H}_0A^\top$}\\
        \end{align*}
        now we have 
        \begin{align*}
            x^{(1)} &= Az^{(1)} + b\\
            &= A\left(z^{(0)} - A^{-1} H_0 \nabla f(x)\right) + b\\
            &= Az^{(0)} - AA^{-1} H_0 \nabla f(x) + b\\
            &= Az^{(0)} + b - H_0 \nabla f(x)\\
            &= x^{(0)} - H_0 \nabla f(x)\\
        \end{align*}
        {\bf Induction hypothesis:} assume $x^{(k)} = x^{(k-1)} - H_{(k-1)}\nabla f(x)$ 
        We want to show it is also the case for $x^{(k+1)}$.

        {\bf Inductive step:}
        we have such identity
        \begin{align*}
            s^{(k)} &= x^{(k+1)} - x^{(k)} = A(z^{(k+1)} - z^{(k)}) = A\tilde{s}^{(k)} \tag*{where $z^{(k+1)} - z^{(k)}$}\\
            y^{(k)} &= \nabla g(z^{(k+1)}) - \nabla g(z^{(k)}) = A^\top \nabla f(x^{(k+1)}) - A^\top \nabla f(x^{(k)}) = A^\top \tilde{y}^{(k)} \tag*{where $\nabla f(x^{(k+1)}) - \nabla f(x^{(k)})$}
        \end{align*}
        We can try to show by induction that this identity holds: $H_k = A\tilde{H}_kA^\top$.
        and the update formular for $z^{(k+1)}$ as such
        \begin{align*}
            z^{(k+1)} &= z^{(k)} - \tilde{H}_k\nabla g(z^{(k)})\\
            &= z^{(k)} - \tilde{H}_k A^\top \nabla f(x)\\
            &= z^{(k)} - A^{-1} H_k \nabla f(x)\\
        \end{align*}
        Then
        \begin{align*}
            x^{(k+1)} &= Az^{(k+1)} + b\\
            &= A\left(z^{(k)} - A^{-1} H_k \nabla f(x)\right) + b\\
            &= Az^{(k)} - AA^{-1} H_k \nabla f(x) + b\\
            &= Az^{(k)} + b - H_k \nabla f(x)\\
            &= x^{(k)} - H_k \nabla f(x)\\
        \end{align*}
    \end{homeworkSubsection}
\end{homeworkProblem}
\end{document}
