\documentclass{article}

% Packages forked from the original template
\usepackage{fancyhdr}
\usepackage{extramarks}
\usepackage{amsmath}
\usepackage{amsthm}
\usepackage{amsfonts}
\usepackage{tikz}
\usepackage[plain]{algorithm}
\usepackage{algpseudocode}

% Extra packages
\usepackage{changepage} % adjustwidth environment
\usepackage{hyperref} % href

\usetikzlibrary{automata,positioning}

%
% Basic Document Settings
%

\topmargin=-0.45in
\evensidemargin=0in
\oddsidemargin=0in
\textwidth=6.5in
\textheight=9.0in
\headsep=0.25in

\linespread{1.1}

\pagestyle{fancy}
\lhead{\hmwkClass: \hmwkTitle}
\rhead{\firstxmark}
\lfoot{\lastxmark}
\cfoot{\thepage}

\renewcommand\headrulewidth{0.4pt}
\renewcommand\footrulewidth{0.4pt}

\setlength\parindent{0pt}

%
% Create Problem Sections
%

\newcommand{\enterProblemHeader}[1]{
    \nobreak\extramarks{}{Exercise \arabic{#1} continued on next page\ldots}\nobreak{}
    \nobreak\extramarks{Exercise \arabic{#1} (continued)}{Exercise \arabic{#1} continued on next page\ldots}\nobreak{}
}

\newcommand{\exitProblemHeader}[1]{
    \nobreak\extramarks{Exercise \arabic{#1} (continued)}{Exercise \arabic{#1} continued on next page\ldots}\nobreak{}
    \stepcounter{#1}
    \nobreak\extramarks{Exercise \arabic{#1}}{}\nobreak{}
}

\setcounter{secnumdepth}{0}
\newcounter{partCounter}
\newcounter{homeworkProblemCounter}
\setcounter{homeworkProblemCounter}{1}
\nobreak\extramarks{Exercise \arabic{homeworkProblemCounter}}{}\nobreak{}

%
% Homework Problem Environment
%
% This environment takes an optional argument. When given, it will adjust the
% problem counter. This is useful for when the problems given for your
% assignment aren't sequential. See the last 3 problems of this template for an
% example.
%
\newenvironment{homeworkProblem}[1][-1]{
    \ifnum#1>0
        \setcounter{homeworkProblemCounter}{#1}
    \fi
    \section{Exercise \arabic{homeworkProblemCounter}}
    \setcounter{partCounter}{1}
    \enterProblemHeader{homeworkProblemCounter}
}{
    \exitProblemHeader{homeworkProblemCounter}
}

%
% Homework Details
%   - Title
%   - Due date
%   - Class
%   - Section/Time
%   - Instructor
%   - Author
%

\newcommand{\hmwkTitle}{Assignment 1}
\newcommand{\hmwkDueDate}{April 30, 2014}
\newcommand{\hmwkClass}{Continuous Optimization}
\newcommand{\hmwkAuthorName}{ \textbf{Honglu Ma} \and \textbf{Hiroyasu Akada} \and \textbf{Mathivathana Ayyappan}}

%
% Title Page
%

\title{
    \vspace{2in}
    \textmd{\textbf{\hmwkClass:\ \hmwkTitle}}\\
    \normalsize\vspace{0.1in}\small{Due\ on\ \hmwkDueDate}\\
    \vspace{3in}
}

\author{\hmwkAuthorName}
\date{}

\renewcommand{\part}[1]{\textbf{\large Part \Alph{partCounter}}\stepcounter{partCounter}\\}

%
% Various Helper Commands
%

% Useful for algorithms
\newcommand{\alg}[1]{\textsc{\bfseries \footnotesize #1}}

% For derivatives
\newcommand{\deriv}[1]{\frac{\mathrm{d}}{\mathrm{d}x} (#1)}

% For partial derivatives
\newcommand{\pderiv}[2]{\frac{\partial}{\partial #1} (#2)}

% Integral dx
\newcommand{\dx}{\mathrm{d}x}

% Alias for the Solution section header
\newcommand{\solution}{\textbf{\large Solution}}

% Probability commands: Expectation, Variance, Covariance, Bias
\newcommand{\E}{\mathrm{E}}
\newcommand{\Var}{\mathrm{Var}}
\newcommand{\Cov}{\mathrm{Cov}}
\newcommand{\Bias}{\mathrm{Bias}}

% Norm
\newcommand{\norm}[1]{\left\lVert#1\right\rVert}

% Margined Homework Subsection
\newenvironment{homeworkSubsection}[1]{%
    \subsection*{#1}%
    \begin{adjustwidth}{2.5em}{0pt}%
}{%
    \end{adjustwidth}%
}

\begin{document}

\maketitle

\pagebreak

\begin{homeworkProblem}
    \begin{homeworkSubsection}{(a)}
        \textbf{Claim: } $ lim_{k \to \infty} x^{(k)} = \frac{1}{30}$ \\
        \textbf{Proof: } \\
        Let $\epsilon > 0$ be given. Choose $N = \max\{\frac{1}{90\epsilon}, 8\}$.
        Assume $n > N$. We have 
        \[
            n > N
            \Rightarrow n > 9 > \root{3}\of{600}
            \Rightarrow n^3 > 600
            \Rightarrow 5n^3 > 3000
            \Rightarrow 10n^3 - 5n^3 > 3000
            \Rightarrow 3000 + 5n^3 < 10n^3
        \]
        and obviously
        \[
            900n^4 > 150n^3 + 900n^4
        \]
        To check the validity of the limit we need to show $\left|x^{(n)} - x^\star\right| < \epsilon$ where $x^\star = \frac{1}{30}$.
        \begin{align*}
            \left|\frac{n^4 - 100}{5n^3 + 30k^4} - \frac{1}{30}\right| 
            &= \left|\frac{30n^4 - 3000 - 5n^3 - 30k^4}{150n^3 + 900n^4}\right| \\
            &= \left|\frac{-3000 - 5n^3}{150n^3 + 900n^4}\right|\\
            &= \frac{3000 + 5n^3}{150n^3 + 900n^4}\\
            &< \frac{10n^3}{900n^4} = \frac{1}{90n} \tag*{(by the inequalities above)}\\
            &< \frac{1}{90N}\\
            &< \frac{1}{\frac{90}{90\epsilon}} = \epsilon
        \end{align*}
    \end{homeworkSubsection}
    \begin{homeworkSubsection}{(b)}
        We have the following accumulation points:
        \[
            x^{(8n+1)} = \begin{pmatrix}
                \frac{\sqrt{2}}{2} & -\frac{\sqrt{2}}{2}\\
                \frac{\sqrt{2}}{2} & \frac{\sqrt{2}}{2}
            \end{pmatrix}
            \begin{pmatrix}
                1\\
                0
            \end{pmatrix}
            = \begin{pmatrix}
                \frac{\sqrt{2}}{2}\\
                \frac{\sqrt{2}}{2}
            \end{pmatrix}, \quad
            x^{(8n+2)} = \begin{pmatrix}
                0\\
                1
            \end{pmatrix}, \quad
            x^{(8n+3)} = \begin{pmatrix}
                -\frac{\sqrt{2}}{2}\\
                \frac{\sqrt{2}}{2}
            \end{pmatrix}, \quad
            x^{(8n+4)} = \begin{pmatrix}
                -1\\
                0
            \end{pmatrix}
        \]
        \[
            x^{(8n+5)} = \begin{pmatrix}
                -\frac{\sqrt{2}}{2}\\
                -\frac{\sqrt{2}}{2}
            \end{pmatrix}, \quad
            x^{(8n+6)} = \begin{pmatrix}
                0\\
                -1
            \end{pmatrix}, \quad
            x^{(8n+7)} = \begin{pmatrix}
                \frac{\sqrt{2}}{2}\\
                -\frac{\sqrt{2}}{2}
            \end{pmatrix}, \quad
            x^{(8n)} = \begin{pmatrix}
                1\\
                0
            \end{pmatrix}, \quad
            n \in \mathbb{N}
        \]
        We can prove by showing that $x^{(k)}$ is not a cauchy sequence thus does not converges (Proposition A.5, Lecture Script)
        i.e. $\exists \epsilon > 0$ such that 
        $\forall N \in \mathbb{N}$, $\exists n, m > N$ such that $\norm{x^{(n)} - x^{(m)}} \geq \epsilon$.\\
        \textbf{Proof: } \\
        Let $\epsilon = 1$, for all $N \in \mathbb{N}$, choose $n = 8N$ and $m = 8N+4$. We have
        \[
            \norm{x^{(8N)} - x^{(8N+4)}} = \norm{\begin{pmatrix}
                1\\
                0
            \end{pmatrix}-\begin{pmatrix}
                -1\\
                0
            \end{pmatrix}
            }=2 \geq 1 = \epsilon
        \]
    \end{homeworkSubsection}
\end{homeworkProblem}
\begin{homeworkProblem}
    \begin{homeworkSubsection}{(a)}
        The interior of a set $C$ is defined as the union of all open sets contained in $C$.
        The closure of a set $C$ is the set $C$ together with all of its limit points.
        \begin{enumerate}
            \item[(i)]Let $C = \mathbb{R}$. 
            We have $\mathrm{int}(C) = \mathbb{R}$ and $\mathrm{cl}(C) = \mathbb{R}$.
            It is obvious that $\mathrm{int}(\mathrm{cl}(C)) = \mathrm{int}(C) = \mathbb{R}$.
            \item[(ii)]Let $C = \mathbb{Q}$. 
            We have $\mathrm{int}(C) = \emptyset$ and $\mathrm{cl}(C) = \mathbb{R}$.
            This is because every open interval in $\mathbb{R}$ contains irrational numbers.
            Thus $\mathrm{int}(\mathrm{cl}(C)) = \mathrm{int}(\mathbb{R}) = \mathbb{R} \neq \mathrm{int}(C) = \emptyset$.
        \end{enumerate}
    \end{homeworkSubsection}
    \begin{homeworkSubsection}{(b)}
        Consider function $f(x) = \sqrt{||x||_1},\,x \in \mathbb{R}^2$ has sublevel sets $\{x \in \mathbb{R}^2 \mid f(x) \leq \alpha\}$.
        The function can also be written as $f(x) = \sqrt{|x_1| + |x_2|},\,x = (x_1, x_2)$.

        Pick two elements $x, y$ from the sublevel set of $\alpha$ such that $f(x) \leq \alpha$ and $f(y) \leq \alpha$. 
        We have
        \[
            f(x) = \sqrt{|x_1| + |x_2|} \leq \alpha \Rightarrow |x_1| + |x_2| \leq \alpha^2
            \text{ and similarly } |y_1| + |y_2| \leq \alpha^2
        \]
        Now take a point $z := \lambda x + (1-\lambda)y,\,\lambda \in [0, 1]$.
        Also $z_1 = \lambda x_1 + (1-\lambda)y_1,\, z_2 = \lambda x_2 + (1-\lambda)y_2$
        \begin{align*}
            |z_1| + |z_2|
            &= |\lambda x_1 + (1-\lambda)y_1| + |\lambda x_2 + (1-\lambda)y_2|\\
            &\leq \lambda |x_1| + (1-\lambda)|y_1| + \lambda |x_2| + (1-\lambda)|y_2|\\
            &\leq \lambda (|x_1| + |x_2|) + (1-\lambda)(|y_1| + |y_2|)\\
            &\leq \lambda \alpha^2 + (1-\lambda)\alpha^2 = \alpha^2
        \end{align*}
        thus $\{x \in \mathbb{R}^2 \mid f(x) \leq \alpha\}$ is a convex set.\\
        Now we need to show that $f$ is not convex.
        Let $x = (0, 0),\,y = (1, 0)$ and $\lambda = \frac{1}{2}$.
        We have $f(x) = 0,\,f(y) = 1$ and
        \[
            f(\lambda x + (1-\lambda)y) = \sqrt{\frac{1}{2}} > \lambda f(x) + (1-\frac{1}{2})f(y) = \frac{1}{2}
        \]
        Check \href{https://www.geogebra.org/3d/aqnq3xcm}{here} for visualization of the sublevel sets.
    \end{homeworkSubsection}
    \begin{homeworkSubsection}{(c)}
        Let $v_{N_A}$ be $v$'s projection onto the null space and $\lambda \in \mathbb{R}^m$.
        We know that $v$ can be decomposite into
        \begin{align*}
            v = A^\top\lambda + v_{N_A}
            &\Rightarrow Av = A(A^\top\lambda + v_{N_A})\\
            &\Rightarrow Av = AA^\top\lambda + 0\\
            &\Rightarrow Av = AA^\top\lambda\\
            &\Rightarrow (AA^\top)^{-1}Av = \lambda\\
        \end{align*}
        Replace $\lambda$ in the original equation we get
        \begin{align*}
            v = A^\top(AA^\top)^{-1}Av + v_{N_A}
            &\Rightarrow v - A^\top(AA^\top)^{-1}Av = v_{N_A}\\
            &\Rightarrow (I - A^\top(AA^\top)^{-1}A)v = v_{N_A}\\
        \end{align*}
    \end{homeworkSubsection}
    \begin{homeworkSubsection}{(d)}
        \begin{enumerate}
            \item[(i)] Take $v \in \{x\in E \mid f(x) \geq c\}$ we have 
            \[
                f(v) \geq c > c - \frac{1}{k} \text{ for all } k > 1
            \]
            This shows that $v \in \{x\in E \mid f(x) > c - \frac{1}{k}\}$ for all $k > 1$.

            \item[(ii)] Take $v \in \{x\in E \mid f(x) > c\}$. Let $f(v) = c + \frac{1}{k}$.

            We can always pick a $k'$ such that 
            $k' \geq k$ so that $f(v) = c + \frac{1}{k} \geq c + \frac{1}{k'}$.

            This shows that $\exists k' > 1$ such that $v \in \{x\in E \mid f(x) \geq c + \frac{1}{k'}\}$.
          \end{enumerate}
    \end{homeworkSubsection}
\end{homeworkProblem}
\begin{homeworkProblem}
    \begin{homeworkSubsection}{(i)}
        \textbf{Proof: } Assume that $\exists A_1, A_2: \mathbb{R}^n \to \mathbb{R}^n$ such that
        \[
            \lim_{h \to 0} \frac{||f(x + h) - f(x) - A_1h||}{||h||} = 0
            \lim_{h \to 0} \frac{||f(x + h) - f(x) - A_2h||}{||h||} = 0
        \]
        Subtracting the two equations:
        \begin{align*}
            &\lim_{h \to 0} \frac{||f(x + h) - f(x) - A_1h||}{||h||} - \lim_{h \to 0} \frac{||f(x + h) - f(x) - A_2h||}{||h||} = 0\\
            &\Rightarrow \lim_{h \to 0} \frac{||f(x + h) - f(x) - A_1h - f(x + h) + f(x) + A_2h||}{||h||} = 0\\
            &\Rightarrow \lim_{h \to 0} \frac{||A_2h - A_1h||}{||h||} = 0\\
            &\Rightarrow \lim_{h \to 0} \frac{||(A_2- A_1)h||}{||h||} = 0 \\
            &\Rightarrow \lim_{\alpha \to 0} \frac{||(A_2- A_1)\alpha u||}{||\alpha u||} = 0
            \tag*{let $\alpha u = h$, $\alpha \in \mathbb{R}$ and $u$ is an unit vector}\\
            &\Rightarrow \lim_{\alpha \to 0} \frac{\alpha^2||(A_2- A_1)u||}{\alpha^2||u||} = 0\\
            &\Rightarrow \lim_{\alpha \to 0} \frac{||(A_2- A_1)u||}{||u||} = 0\\
            &\Rightarrow \frac{||(A_2- A_1)u||}{||u||} = 0\\
            &\Rightarrow ||(A_2- A_1)u|| = 0\\
            &\Rightarrow A_2=A_1
            \tag*{linear transformation is unique}\\
        \end{align*}
    \end{homeworkSubsection}
    \begin{homeworkSubsection}{(ii)}
        
    \end{homeworkSubsection}
    \begin{homeworkSubsection}{(iii)}
        \[
            A = \begin{pmatrix}
                \frac{\partial f_1}{\partial x_1} & \cdots & \frac{\partial f_1}{\partial x_n} \\
                \vdots & \ddots & \vdots \\
                \frac{\partial f_m}{\partial x_1} & \cdots & \frac{\partial f_m}{\partial x_n}
            \end{pmatrix}
        \]
    \end{homeworkSubsection}
\end{homeworkProblem}
\begin{homeworkProblem}
    \begin{homeworkSubsection}{(a)}
        The function can be rewritten as
        \[
            f(u) = \frac{1}{2}||u - c||^2 + \frac{\mu}{2}||Au||^2
        \]
        where $u \in \mathbb{R}^n$, $A \in \mathbb{R}^{N \times N}$ and each element at row $i < N$ and column $j$ is defined as
        \[
            A_{ij} := \begin{cases}
                -1 & \text{if } i = j\\
                1 & \text{if } i = j+1\\
                0 & \text{otherwise}
            \end{cases}
        \]
        and its last row is defined as $A_{Nj} := 0$
    \end{homeworkSubsection}
    \begin{homeworkSubsection}{(b)}
        For each entry of $\nabla f(u)$ we have
        \[
            \frac{\partial f}{\partial u_i} = \begin{cases}
                u_i - c_i + \mu (u_i - u_{i+1}) & \text{if } i = 1\\
                u_i - c_i + \mu (-u_{i-1} + 2u_i -u_{i+1}) & \text{if } 1 < i < n\\
                u_i - c_i + \mu (-u_{i-1} + u_i) & \text{if } i = n
            \end{cases}
        \]
    \end{homeworkSubsection}
    \begin{homeworkSubsection}{(c)}
        \[
            \nabla f(u) = u - c + \mu A^\top Au
        \]
        which can be verified since each element of $A^\top A$ can be expressed as
        \begin{align*}
            (A^\top A)_{1j} &= \begin{cases}
                1 & \text{if } j = 1\\
                -1 & \text{if } j = 2\\
                0 & \text{otherwise}
            \end{cases}\\
            (A^\top A)_{ij} &= \begin{cases}
                -1 & \text{if } j = i-1\\
                2 & \text{if } j = i\\
                -1 & \text{if } j = i+1\\
                0 & \text{otherwise}
            \end{cases}, 2 \leq i \leq N-2\\
            (A^\top A)_{Nj} &= \begin{cases}
                -1 & \text{if } j = N-1\\
                1 & \text{if } j = N\\
                0 & \text{otherwise}
            \end{cases}
        \end{align*}
    \end{homeworkSubsection}
    \begin{homeworkSubsection}{(d)}
        \begin{align*}
            \nabla f(u) = 0
            &\Rightarrow u - c + \mu A^\top Au = 0\\
            &\Rightarrow (\mu A^\top A + I)u = c\\
        \end{align*}
        $\mu A^\top A + I$ is symmetric. if $\det(\mu A^\top A + I) \ne 0 $ then $\mu A^\top A + I$ is invertible
        and we can write
        \[
            u = (\mu A^\top A + I)^-1c
        \]
    \end{homeworkSubsection}
    \begin{homeworkSubsection}{(e)}
        Argue that if $\mu A^\top A + I$ is invertible then the solution is unique.
    \end{homeworkSubsection}
\end{homeworkProblem}

\end{document}
