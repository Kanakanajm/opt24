\documentclass{article}

% Packages forked from the original template
\usepackage{fancyhdr}
\usepackage{extramarks}
\usepackage{amsmath}
\usepackage{amsthm}
\usepackage{amsfonts}
\usepackage{tikz}
\usepackage[plain]{algorithm}
\usepackage{algpseudocode}

% Extra packages
\usepackage{changepage} % adjustwidth environment

\usetikzlibrary{automata,positioning}

%
% Basic Document Settings
%

\topmargin=-0.45in
\evensidemargin=0in
\oddsidemargin=0in
\textwidth=6.5in
\textheight=9.0in
\headsep=0.25in

\linespread{1.1}

\pagestyle{fancy}
\lhead{\hmwkAuthorName}
\chead{\hmwkClass: \hmwkTitle}
\rhead{\firstxmark}
\lfoot{\lastxmark}
\cfoot{\thepage}

\renewcommand\headrulewidth{0.4pt}
\renewcommand\footrulewidth{0.4pt}

\setlength\parindent{0pt}

%
% Create Problem Sections
%

\newcommand{\enterProblemHeader}[1]{
    \nobreak\extramarks{}{Exercise \arabic{#1} continued on next page\ldots}\nobreak{}
    \nobreak\extramarks{Exercise \arabic{#1} (continued)}{Exercise \arabic{#1} continued on next page\ldots}\nobreak{}
}

\newcommand{\exitProblemHeader}[1]{
    \nobreak\extramarks{Exercise \arabic{#1} (continued)}{Exercise \arabic{#1} continued on next page\ldots}\nobreak{}
    \stepcounter{#1}
    \nobreak\extramarks{Exercise \arabic{#1}}{}\nobreak{}
}

\setcounter{secnumdepth}{0}
\newcounter{partCounter}
\newcounter{homeworkProblemCounter}
\setcounter{homeworkProblemCounter}{1}
\nobreak\extramarks{Exercise \arabic{homeworkProblemCounter}}{}\nobreak{}

%
% Homework Problem Environment
%
% This environment takes an optional argument. When given, it will adjust the
% problem counter. This is useful for when the problems given for your
% assignment aren't sequential. See the last 3 problems of this template for an
% example.
%
\newenvironment{homeworkProblem}[1][-1]{
    \ifnum#1>0
        \setcounter{homeworkProblemCounter}{#1}
    \fi
    \section{Exercise \arabic{homeworkProblemCounter}}
    \setcounter{partCounter}{1}
    \enterProblemHeader{homeworkProblemCounter}
}{
    \exitProblemHeader{homeworkProblemCounter}
}

%
% Homework Details
%   - Title
%   - Due date
%   - Class
%   - Section/Time
%   - Instructor
%   - Author
%

\newcommand{\hmwkTitle}{Assignment 1}
\newcommand{\hmwkDueDate}{April 30, 2014}
\newcommand{\hmwkClass}{Continuous Optimization}
\newcommand{\hmwkAuthorName}{ \textbf{Hiroyasu Akada} \and \textbf{Honglu Ma}}

%
% Title Page
%

\title{
    \vspace{2in}
    \textmd{\textbf{\hmwkClass:\ \hmwkTitle}}\\
    \normalsize\vspace{0.1in}\small{Due\ on\ \hmwkDueDate}\\
    \vspace{3in}
}

\author{\hmwkAuthorName}
\date{}

\renewcommand{\part}[1]{\textbf{\large Part \Alph{partCounter}}\stepcounter{partCounter}\\}

%
% Various Helper Commands
%

% Useful for algorithms
\newcommand{\alg}[1]{\textsc{\bfseries \footnotesize #1}}

% For derivatives
\newcommand{\deriv}[1]{\frac{\mathrm{d}}{\mathrm{d}x} (#1)}

% For partial derivatives
\newcommand{\pderiv}[2]{\frac{\partial}{\partial #1} (#2)}

% Integral dx
\newcommand{\dx}{\mathrm{d}x}

% Alias for the Solution section header
\newcommand{\solution}{\textbf{\large Solution}}

% Probability commands: Expectation, Variance, Covariance, Bias
\newcommand{\E}{\mathrm{E}}
\newcommand{\Var}{\mathrm{Var}}
\newcommand{\Cov}{\mathrm{Cov}}
\newcommand{\Bias}{\mathrm{Bias}}

% Norm
\newcommand{\norm}[1]{\left\lVert#1\right\rVert}

% Margined Homework Subsection
\newenvironment{homeworkSubsection}[1]{%
    \subsection*{#1}%
    \begin{adjustwidth}{2.5em}{0pt}%
}{%
    \end{adjustwidth}%
}

\begin{document}

\maketitle

\pagebreak

\begin{homeworkProblem}
    \begin{homeworkSubsection}{(a)}
        \textbf{Claim: } $ lim_{k \to \infty} x^{(k)} = \frac{1}{30}$ \\
        \textbf{Proof: } \\
        Let $\epsilon > 0$ be given. Choose $N = \max\{\frac{1}{90\epsilon}, 8\}$.
        Assume $n > N$. We have 
        \[
            n > N
            \Rightarrow n > 9 > \root{3}\of{600}
            \Rightarrow n^3 > 600
            \Rightarrow 5n^3 > 3000
            \Rightarrow 10n^3 - 5n^3 > 3000
            \Rightarrow 3000 + 5n^3 < 10n^3
        \]
        and obviously
        \[
            900n^4 > 150n^3 + 900n^4
        \]
        To check the validity of the limit we need to show $\left|x^{(n)} - x^\star\right| < \epsilon$ where $x^\star = \frac{1}{30}$.
        \begin{align*}
            \left|\frac{n^4 - 100}{5n^3 + 30k^4} - \frac{1}{30}\right| 
            &= \left|\frac{30n^4 - 3000 - 5n^3 - 30k^4}{150n^3 + 900n^4}\right| \\
            &= \left|\frac{-3000 - 5n^3}{150n^3 + 900n^4}\right|\\
            &= \frac{3000 + 5n^3}{150n^3 + 900n^4}\\
            &< \frac{10n^3}{900n^4} = \frac{1}{90n} \tag*{(by the inequalities above)}\\
            &< \frac{1}{90N}\\
            &< \frac{1}{\frac{90}{90\epsilon}} = \epsilon
        \end{align*}
    \end{homeworkSubsection}
    \begin{homeworkSubsection}{(b)}
        We have the sequence as such:
        \[
            x^{(1)} = \begin{pmatrix}
                \frac{\sqrt{2}}{2} & -\frac{\sqrt{2}}{2}\\
                \frac{\sqrt{2}}{2} & \frac{\sqrt{2}}{2}
            \end{pmatrix}
            \begin{pmatrix}
                1\\
                0
            \end{pmatrix}
            = \begin{pmatrix}
                \frac{\sqrt{2}}{2}\\
                \frac{\sqrt{2}}{2}
            \end{pmatrix}, \quad
            x^{(2)} = \begin{pmatrix}
                0\\
                1
            \end{pmatrix}, \quad
            x^{(3)} = \begin{pmatrix}
                -\frac{\sqrt{2}}{2}\\
                \frac{\sqrt{2}}{2}
            \end{pmatrix}, \quad
            x^{(4)} = \begin{pmatrix}
                -1\\
                0
            \end{pmatrix}
        \]
        \[
            x^{(5)} = \begin{pmatrix}
                -\frac{\sqrt{2}}{2}\\
                -\frac{\sqrt{2}}{2}
            \end{pmatrix}, \quad
            x^{(6)} = \begin{pmatrix}
                0\\
                -1
            \end{pmatrix}, \quad
            x^{(7)} = \begin{pmatrix}
                \frac{\sqrt{2}}{2}\\
                -\frac{\sqrt{2}}{2}
            \end{pmatrix}, \quad
            x^{(8)} = \begin{pmatrix}
                1\\
                0
            \end{pmatrix}, \quad\ldots
        \]
        We can prove by showing that $x^{(k)}$ is not a cauchy sequence thus does not converges (Proposition A.5, Lecture Script)
        i.e. $\exists \epsilon > 0$ such that 
        $\forall N \in \mathbb{N}$, $\exists n, m > N$ such that $\norm{x^{(n)} - x^{(m)}} \geq \epsilon$.\\
        \textbf{Proof: } \\
        Let $\epsilon = 1$, for all $N \in \mathbb{N}$, choose $n = 8N$ and $m = 8N+4$. We have
        \[
            \norm{x^{(8N)} - x^{(8N+4)}} = \norm{\begin{pmatrix}
                1\\
                0
            \end{pmatrix}-\begin{pmatrix}
                -1\\
                0
            \end{pmatrix}
            }=2 \geq 1 = \epsilon
        \]
        \textcolor{red}{We need to prove by induction that the sequence is periodic.
        We can determine cluster point by constructing subsequence but how to prove that we have found all of them?
        }
    \end{homeworkSubsection}
\end{homeworkProblem}

\end{document}
