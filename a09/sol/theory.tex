\documentclass{article}

% Packages forked from the original template
\usepackage{fancyhdr}
\usepackage{extramarks}
\usepackage{amsmath}
\usepackage{amsthm}
\usepackage{amsfonts}
\usepackage{tikz}
\usepackage[plain]{algorithm}
\usepackage{algpseudocode}

% Extra packages
\usepackage{changepage} % adjustwidth environment
\usepackage{hyperref} % href
\usepackage{xcolor} % colored text

\usetikzlibrary{automata,positioning}

% Additional commands
\def\eg{\emph{e.g., }}
\def\ie{\emph{i.e., }}
\def\cf{\emph{c.f., }}
\def\etc{\emph{etc. }}
\def\wrt{\emph{w.r.t. }}
\def\etal{\emph{et al. }}

%
% Basic Document Settings
%

\topmargin=-0.45in
\evensidemargin=0in
\oddsidemargin=0in
\textwidth=6.5in
\textheight=9.0in
\headsep=0.25in

\linespread{1.1}

\pagestyle{fancy}
\lhead{\hmwkClass: \hmwkTitle}
\rhead{\firstxmark}
\lfoot{\lastxmark}
\cfoot{\thepage}

\renewcommand\headrulewidth{0.4pt}
\renewcommand\footrulewidth{0.4pt}

\setlength\parindent{0pt}

%
% Create Problem Sections
%

\newcommand{\enterProblemHeader}[1]{
    \nobreak\extramarks{}{Exercise \arabic{#1} continued on next page\ldots}\nobreak{}
    \nobreak\extramarks{Exercise \arabic{#1} (continued)}{Exercise \arabic{#1} continued on next page\ldots}\nobreak{}
}

\newcommand{\exitProblemHeader}[1]{
    \nobreak\extramarks{Exercise \arabic{#1} (continued)}{Exercise \arabic{#1} continued on next page\ldots}\nobreak{}
    \stepcounter{#1}
    \nobreak\extramarks{Exercise \arabic{#1}}{}\nobreak{}
}

\setcounter{secnumdepth}{0}
\newcounter{partCounter}
\newcounter{homeworkProblemCounter}
\setcounter{homeworkProblemCounter}{1}
\nobreak\extramarks{Exercise \arabic{homeworkProblemCounter}}{}\nobreak{}

%
% Homework Problem Environment
%
% This environment takes an optional argument. When given, it will adjust the
% problem counter. This is useful for when the problems given for your
% assignment aren't sequential. See the last 3 problems of this template for an
% example.
%
\newenvironment{homeworkProblem}[1][-1]{
    \ifnum#1>0
        \setcounter{homeworkProblemCounter}{#1}
    \fi
    \section{Exercise \arabic{homeworkProblemCounter}}
    \setcounter{partCounter}{1}
    \enterProblemHeader{homeworkProblemCounter}
}{
    \exitProblemHeader{homeworkProblemCounter}
}

%
% Homework Details
%   - Title
%   - Due date
%   - Class
%   - Section/Time
%   - Instructor
%   - Author
%

\newcommand{\hmwkTitle}{Assignment 9}
\newcommand{\hmwkDueDate}{June 25, 2024}
\newcommand{\hmwkClass}{Continuous Optimization}
\newcommand{\hmwkAuthorName}{ \textbf{Honglu Ma} \and \textbf{Hiroyasu Akada} \and \textbf{Mathivathana Ayyappan}}

%
% Title Page
%

\title{
    \vspace{2in}
    \textmd{\textbf{\hmwkClass:\ \hmwkTitle}}\\
    \normalsize\vspace{0.1in}\small{Due\ on\ \hmwkDueDate}\\
    \vspace{3in}
}

\author{\hmwkAuthorName}
\date{}

\renewcommand{\part}[1]{\textbf{\large Part \Alph{partCounter}}\stepcounter{partCounter}\\}

%
% Various Helper Commands
%

% Useful for algorithms
\newcommand{\alg}[1]{\textsc{\bfseries \footnotesize #1}}

% For derivatives
\newcommand{\deriv}[1]{\frac{\mathrm{d}}{\mathrm{d}x} (#1)}

% For partial derivatives
\newcommand{\pderiv}[2]{\frac{\partial}{\partial #1} (#2)}

% Integral dx
\newcommand{\dx}{\mathrm{d}x}

% Alias for the Solution section header
\newcommand{\solution}{\textbf{\large Solution}}

% Probability commands: Expectation, Variance, Covariance, Bias
\newcommand{\E}{\mathrm{E}}
\newcommand{\Var}{\mathrm{Var}}
\newcommand{\Cov}{\mathrm{Cov}}
\newcommand{\Bias}{\mathrm{Bias}}

% Norm
\newcommand{\norm}[1]{\left\lVert#1\right\rVert}

% Margined Homework Subsection
\newenvironment{homeworkSubsection}[1]{%
    \subsection*{#1}%
    \begin{adjustwidth}{2.5em}{0pt}%
}{%
    \end{adjustwidth}%
}

\begin{document}

\maketitle

\pagebreak

\begin{homeworkProblem}[1]
    The problem of finding the clostest point to another point can be formulated as
    \[
        \min_{x \in \mathbb{R}^3} ||x - p||^2 \quad \text{s.t.} \quad ||x||^2 = 4
    \]    
    while the problem of finding the farthest point can be formulated as
    \[
        \max_{x \in \mathbb{R}^3} ||x - p||^2 \quad \text{s.t.} \quad ||x||^2 = 4
    \]
    which is equivalent to
    \[
        \min_{x \in \mathbb{R}^3} -||x - p||^2 \quad \text{s.t.} \quad ||x||^2 = 4
    \]
    where $p = (2, 4, 2)^\top$
    \begin{homeworkSubsection}{Closest Point}
        We want the constraint levelset to be tangent to the curve of the objective function 
        at the optimal point i.e.
        \[
        \]
        where $f(x) = ||x - p||^2$ and $g(x) = ||x||^2$ satisfying $g(x) = 4$.
        
        After calculating the gradients, we have
        \begin{align*}
            2(x - p) &= 2\lambda x\\
            x - p &= \lambda x\\
            x &= \frac{1}{1 - \lambda} p
        \end{align*}
        $x$ still has to satisfy the constraint $||x||^2 = 4$.
        \begin{align*}
            \frac{1}{(1 - \lambda)^2} ||p||^2 &= 4\\
            \frac{24}{(1 - \lambda)^2} &= 4\\
            \frac{6}{(1 - \lambda)^2} &= 1\\
            1 - \lambda &= \pm \sqrt{6}\\
            \lambda &= 1 \pm \sqrt{6}
        \end{align*}
        We have two solutions for $x$:
        \[
            x = \pm\frac{1}{\sqrt{6}}p
        \]
        The Langrange multiplier method gives us only the stationary points
        and we have to determine the minimum by checking which solution results in a smaller objective function value.
        \[ 
            ||x - p||^2 = \frac{1+\sqrt{6}}{-\sqrt{6}}||p||^2 \text{ or } \frac{1-\sqrt{6}}{\sqrt{6}}||p||^2
        \]
        Apparently, the latter is smaller.

        Thus, the closest point is $-\frac{1}{\sqrt{6}}\begin{pmatrix}
            2\\4\\2
        \end{pmatrix}$.
    \end{homeworkSubsection}
    \begin{homeworkSubsection}{Farthest Point}
        Similarly, we have 
        \begin{align*}
            2(x - p) &= -2\lambda x\\
            x - p &= -\lambda x\\
            x &= \frac{1}{1 + \lambda} p\\
        \end{align*}
        with constraint
        \begin{align*}
            ||\frac{1}{1 + \lambda} p||^2 &= 4\\
            \frac{24}{(1 + \lambda)^2} &= 4\\
            \lambda &= -1 \pm \sqrt{6}
        \end{align*}
        We have two solutions for $x$:
        \[
            x = \pm\frac{1}{\sqrt{6}}p
        \]
        Which is the same as the ones we calculated for the closest point.
        Thus, the farthest point is $\frac{1}{\sqrt{6}}\begin{pmatrix}
            2\\4\\2
        \end{pmatrix}$.
    \end{homeworkSubsection}
    

\end{homeworkProblem}

\begin{homeworkProblem}[2]
    The minimization problem is given by
    \[
        \min_{x \in \mathbb{R}^n} ||x||^2 \quad \text{s.t.} \quad Ax = b
    \]
    It is equivalent to
    \[
        \min_{x \in \mathbb{R}^n} \frac{1}{2} ||x||^2 \quad \text{s.t.} \quad Ax = b
    \]
    Let $f(x) = \frac{1}{2} ||x||^2$ and $c_i(x) = a_i^\top x$
    where $a_i$ is the $i$-th row of $A$.

    The constraint $Ax = b$ can be rewritten as $m$ smaller constraints: $c_i(x) = b_i$ for $i = 1, \ldots, m$.
    
    Using the Lagrange multiplier method, we compose such equation:
    \begin{align*}
        \nabla f(x) &= \sum_{i=1}^m \lambda_i \nabla c_i(x)\\
        x &= \sum_{i=1}^m \lambda_i a_i\\
        x &= A^\top \lambda
    \end{align*}
    where $\lambda_i$ is the Lagrange multiplier for the $i$-th constraint
    and $\lambda$ is a column vector consists of all multipliers.
    
    We also have the constraint level sets:
    \begin{align*}
        Ax &= b\\
        AA^\top \lambda &= b\\
        \lambda &= (AA^\top)^{-1} b\\
        \Rightarrow x &= A^\top (AA^\top)^{-1} b
    \end{align*}
\end{homeworkProblem}

\begin{homeworkProblem}[3]
    We have the following optimization problem:
    \[
        \min_{x \in \mathbb{R}^n} f(x) \quad \text{s.t.} \quad g(x) = 1
    \]
    with $f(x) = c^\top x + \frac{1}{2\tau}||x - \bar{x}||^2$ and $g(x) = \sum_{i=1}^{n} x_i$
    
    where $x_i$ is the $i$-th element of $x$.

    Using the Lagrange multiplier method, we have
    \begin{align*}
        \nabla f(x) &= \lambda \nabla g(x)\\
        c + \frac{1}{\tau}(x - \bar{x}) &= \lambda \mathbf{1}
        \tag*{observe that $\nabla g(x) = \mathbf{1}$ which is a column vector of $1$s}\\
        x &= \tau (\lambda \mathbf{1} - c) + \bar{x}
    \end{align*}
    Plug $x$ back into the constraint $g(x) = 1$, we have
    \begin{align*}
        g(\begin{pmatrix}
            \tau(\lambda - c_i) + \bar{x}_i\\
            \vdots\\
            \tau(\lambda - c_n) + \bar{x}_n
        \end{pmatrix}) = 1\\
        \sum_{i=1}^{n} \tau(\lambda - c_i) + \bar{x}_i = 1\\
        n\tau\lambda - \tau\sum_{i=1}^{n} c_i + \sum_{i=1}^{n} \bar{x}_i = 1\\
        n\tau\lambda - \tau g(c) + g(\bar{x}) = 1\\
        \lambda  = \\
    \end{align*}
    Now that $\lambda$ is determined, we can calculate $x$ with the formula we derived earlier
    \begin{align*}
        x &= \tau (\frac{1 + \tau g(c) - g(\bar{x})}{n\tau} \mathbf{1} - c) + \bar{x}\\
        &= (\frac{1 + \tau g(c) - g(\bar{x})}{n} \mathbf{1} - c) + \bar{x}
    \end{align*}
\end{homeworkProblem}
\begin{homeworkProblem}[4]
    {
        \color{red}
        \begin{homeworkSubsection}{(a)}
            We can express the constrained set for each problem as such:
            \[
                C_k = \{x \in \mathbb{R}^n \mid ||x||^2 = 1 \text{ and } \langle v_i, x\rangle \text{ for all } i = 1, \ldots, k-1\}
            \]
            for $k = 1, \ldots, n$.
            The problem can be thus reformulated as
            \[
                \lambda_i = \min_{x \in C_i} \langle x, Qx \rangle \text{ and } v_i \in \underset{x \in C_i}{\mathrm{argmin}} \langle x, Qx \rangle
            \]
            where $i = 1, \ldots, n$.
            Observe that
            \[
                C_n \subset C_{n-1} \subset \ldots \subset C_1
            \]
            We know that for any set $A$ and $B$, if $A \subset B$, then
            \[
                \inf_{x\in D} f(x) \leq \inf_{x\in C} f(x)
            \]
            which in our case, implies \[
                \lambda_1 \leq \lambda_2 \leq \ldots \leq \lambda_n
            \]
        \end{homeworkSubsection}
        \begin{homeworkSubsection}{(b)}
            \textbf{Proof by Contradiction}
    
            Assume that set $V := \{v_1, v_2, \ldots, v_n\}$ is not a linearly independent set
            i.e. there exists $v_k \in V$ s.t $v_k = \sum_{i\neq k} \mu_iv_i$.
            We also know by the constraint, that $\langle v_i, v_j\rangle = 0\;\forall i, j \in \{1,\ldots,n\}, i\neq j$ 
            Take any $v_j$ from $V$ s.t. $j \neq k$ we have
            \begin{align*}
                \langle v_k, v_j \rangle
                &= \langle \sum_{i\neq k} \mu_iv_i, v_j \rangle\\
                &= \langle \sum_{i\neq k, i\neq j} \mu_iv_i, v_j \rangle + \langle v_j, v_j\rangle\\
                &= 0 + \langle v_j, v_j\rangle\\
                &= \langle v_j, v_j\rangle \neq 0\\
            \end{align*}
        \end{homeworkSubsection}    
    }
    \begin{homeworkSubsection}{(c)}
        We know that
        \[
            \lambda_1 = \min_{x \in \mathbb{R}^n} f(x) \quad \text{s.t.} \quad c_1(x) = 1
        \]
        where $f(x) = \langle x, Qx \rangle$ and $c_1(x) = ||x||^2$.
        
        Apply the Lagrange multiplier method, we have
        \begin{align*}
            \nabla f(x) &=  \tilde{\lambda}_1\nabla c_1(x)\\
            Qx &= \tilde{\lambda}_1 x\\
        \end{align*}
        with Lagrange multiplier $\tilde{\lambda}_1$ and constraint $c_1(x) = 1$.
        
        One can interpret this as $\tilde{\lambda}_1$ being the eigenvalue of $Q$ and $x$ being the eigenvector.
    \end{homeworkSubsection}
\end{homeworkProblem}
\begin{homeworkProblem}[5]
    \begin{homeworkSubsection}{(a)}
        We can find the projection of $\bar{x}$ onto set 
        $C = \{x \in \mathbb{R}^n \mid \langle a, x\rangle \leq b \}$
        by solving a constrained optimization problem:
        \[
            \min_{x \in \mathbb{R}^n} ||x - \bar{x}||^2 \quad \text{s.t.} \quad \langle a, x \rangle = b
        \]
        We have the following two conditions:
        \begin{enumerate}
            \item $\hat{x} \in C$
            \item $\hat{x} \notin C$
        \end{enumerate}
        By Fermat's Rule, we know that $-\nabla f(\hat{x}) = N_C(\hat{x})$.

        If $\hat{x} \in C$, then $N_C(\hat{x}) = \{0\}$.
        We have 
        \begin{align*}
            -\nabla f(\hat{x}) &= 0 \\
            \bar{x} - \hat{x} &= 0 \\
            \hat{x} &= \bar{x}
        \end{align*}

        If $\hat{x} \notin C$, then $N_C(\hat{x}) = \{\hat{x} + t\frac{a}{||a||}\mid t \geq 0\}$. We have
        \begin{align*}
            -\nabla f(\hat{x}) &= \hat{x} + t\frac{a}{||a||}\\
            \bar{x} - \hat{x} &= \hat{x} + t\frac{a}{||a||}\\
            \hat{x} &= \frac{1}{2}(\bar{x} - t\frac{a}{||a||})
        \end{align*}
        We also know that $\hat{x} \in C$ i.e. 
        \begin{align*}
            \langle a, \hat{x} \rangle &= b\\
            \langle a, \frac{1}{2}(\bar{x} - t\frac{a}{||a||}) \rangle &= b\\
            \langle a, \bar{x} \rangle - t||a|| &= 2b\\
            t &= \frac{\langle a, \bar{x} \rangle - 2b}{||a||}
        \end{align*}
        Plug $t$ back into the formula of $\hat{x}$, we have
        \[
            \hat{x} = \bar{x} - \frac{\langle a, \bar{x} \rangle}{||a||^2}a
        \]
        In both condition $\hat{x}$ is expressed as a function of $\bar{x}$
        which shows that the projection is singleton.
    \end{homeworkSubsection}
\end{homeworkProblem}
\end{document}
